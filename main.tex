% Template:     Informe/Reporte LaTeX
% Documento:    Archivo principal
% Versión:      5.4.7 (25/06/2018)
% Codificación: UTF-8
%
% Autor: Pablo Pizarro R. @ppizarror
%        Facultad de Ciencias Físicas y Matemáticas
%        Universidad de Chile
%        pablo.pizarro@ing.uchile.cl, ppizarror.com
%
% Manual template: [http://latex.ppizarror.com/Template-Informe/]
% Licencia MIT:    [https://opensource.org/licenses/MIT/]

% CREACIÓN DEL DOCUMENTO
\documentclass[letterpaper,11pt]{article} % Articulo tamaño carta, 11pt
\usepackage[utf8]{inputenc} % Codificación UTF-8

% INFORMACIÓN DEL DOCUMENTO
\def\titulodelinforme {Tarea 2}
\def\temaatratar {5G e IOT}

\def\autordeldocumento {Nombre del autor}
\def\nombredelcurso {Regulación y Competencia del Sector de Telecomunicaciones}
\def\codigodelcurso {EL-6024}

\def\nombreuniversidad {Universidad de Chile}
\def\nombrefacultad {Facultad de Ciencias Físicas y Matemáticas}
\def\departamentouniversidad {Departamento de la Universidad}
\def\imagendepartamento {departamentos/fcfm}
\def\imagendepartamentoescala {0.2}
\def\localizacionuniversidad {Santiago, Chile}

% INTEGRANTES, PROFESORES Y FECHAS
\def\tablaintegrantes {
\begin{tabular}{ll}
	Integrantes:
	& \begin{tabular}[t]{@{}l@{}}
		Integrante 1 \\
		Integrante 2
	\end{tabular} \\
	Profesor:
	& \begin{tabular}[t]{@{}l@{}}
		Profesor 1
	\end{tabular} \\
	Auxiliares:
	& \begin{tabular}[t]{@{}l@{}}
		Auxiliar 1 \\
		Auxiliar 2
	\end{tabular} \\
	Ayudantes:
	& \begin{tabular}[t]{@{}l@{}}
		Ayudante 1 \\
		Ayudante 2
	\end{tabular} \\
	\multicolumn{2}{l}{Ayudante del laboratorio: Ayudante 1} \\
	& \\
	\multicolumn{2}{l}{Fecha de realización: \today} \\
	\multicolumn{2}{l}{Fecha de entrega: \today} \\
	\multicolumn{2}{l}{\localizacionuniversidad}
\end{tabular}}{
}

% CONFIGURACIONES
\input{lib/config}

% IMPORTACIÓN DE LIBRERÍAS
\input{lib/env/imports}

% IMPORTACIÓN DE FUNCIONES Y ENTORNOS
\input{lib/cmd/all}

% IMPORTACIÓN DE ESTILOS
\input{lib/style/all}

% CONFIGURACIÓN INICIAL DEL DOCUMENTO
\input{lib/cfg/init}

% INICIO DE LAS PÁGINAS
\begin{document}

% PORTADA
\input{lib/page/portrait}

% CONFIGURACIÓN DE PÁGINA Y ENCABEZADOS
\input{lib/cfg/page}

% RESUMEN O ABSTRACT

% TABLA DE CONTENIDOS - ÍNDICE
\input{lib/page/index} % Índice, se puede borrar

% CONFIGURACIONES FINALES
\input{lib/cfg/final}

% ======================= INICIO DEL DOCUMENTO =======================

Luego del fallo de la Corte Suprema donde se penalizaba a las 3 grandes empresas de telecomunicaciones del pais (Entel, Movistar y Claro) por haberse adjudicado bloques correspondientes a la banda 700MHz de manera anticompetitiva no siguiendo la pauta de los limites impuestos por la SUBTel a cada una de las empresas. 

De acuerdo a una de las disposiciones del fallo de la Corte Suprema, se estableció la posibilidad de que la SUBTEL revise los limites actuales del espectro, previo a una consulta al TDLC. 








\input{REFS.tex}% Ejemplo, se puede borrar

% FIN DEL DOCUMENTO
\end{document}
